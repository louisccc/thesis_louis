\section{Introduction}
\label{c:intro}

% 起: IoT的蓬勃發展和其衍生出的Challenges 

Internet of Things (IoT for short) envisions a future that a large number of real-world objects will be integrated through the Internet. It aims to turn high-level interactions with physical world into a matter as simple as interacting with virtual world today\cite{Teixeira2013}. Thanks to the advancement in sensor technology and decreased sensor production cost, the growth of sensor deployments has increased over past five years. It was also estimated by European Commission that we would have 50 to 100 billions devices connected through the IoT network\cite{Sundmaeker2010}. As the number of sensors and actuators in an IoT network grows to millions and even billions, interoperability, scalability and flexibility challenges arise\cite{Teixeira2013}, and therefore make IoT development more costly and more difficult. 

% 承: 為了解決這些IoT所衍生出的Challenge, 有哪些solution已經提出來解決什麼

Most sensors and actuators are developed on different platform and are connected to different network. This increases the difficulty to develop on IoT network. To address these challenges in IoT, many researchers proposed middleware solutions. Middleware refers to the software and tools that can help hide the complexity and heterogeneity of the underlying hardware and network platforms, ease the management of system resources, and increase the predictability of application execution\cite{Wang2008}. Sensors and actuators can communicate through different network protocols and different platforms with middleware support. Middleware also needs to provide programming abstraction as an interface for IoT programmers. Many works designed their middleware solutions in a service-oriented style by abstracting sensors, actuators, computation elements as a software component\cite{Barr2002,Fok,Hughes}. These software service components can interact with other service components by exposing their functional properties as interface for the IoT network. 

WuKong is such an intelligent reconfigurable service-oriented IoT middleware that aims to solve IoT challenges and provide a useful drag-and-drop programming tools for IoT programmers to make flow-based programs. Mapping module in WuKong will help programmers to match available physical resources in environment for their flow-based programs. The mapping could have large impact on the performance of application. Some IoT middleware provide programmer to select their mapping relation manually. Some IoT middleware will try to find appropriate physical resource for one-to-one matching after service discovery.


% 轉: 但還是沒有處理的很好的部分 Programming abstraction沒有針對Non-WSN experts去處理


Making an IoT application still requires a great understanding of wireless sensor network. Non-WSN expert such as meteorological researcher, chemist, farmer, knows little about the difference between two temperature sensors, he/she only cares about the sensing factors required to derive their formula instead. For example, for some applications we need temperature sensors with high responsiveness, for some other applications low responsiveness is okay since we care more about high precision, for some applications we need one that can sense in rigid environment. The mapping could have large impact on the performance of application. Some IoT middleware provide programmer to select their mapping relation manually. Some IoT middleware will try to find the most appropriate physical resource for one-to-one matching after service discovery via QoS consideration. 

However, mapping one service component to one physical resource requires programmer to know about low-level details which may not always happen in practice. For example, a meteorological researcher wants to measure air pressure which could either be inferred from temperature and humidity in air or directly be acquired by air pressure sensor. The meteorological researcher wants air pressure data, but has no intereset about how it was acquired in details. Therefore, higher level of programming abstraction is needed for non-WSN expert programmer to easier develop their IoT applications. 


% 合: 所以這篇paper提出的model要解決什麼

Ideally, middleware support for IoT application development should allow programmer make their application without caring about low-level implementation detail. By simply providing their high-level requirements and their logical design of the application, IoT middleware should help them to find a set of appropriate physical resources to compose their application in IoT system with intelligence. The goal of this paper is to present a user-oriented mapping optimization scheme to make every service components can find its appropriate mapping with maximum objective function automatically by considering rich QoS parameters defined by developer in advance. So that when programming an IoT application, programmers can just focus on their logical flow-based program design instead of struggling with low-level hardware details. 

The rest of the paper is structured as follows: In section II, we will briefly review the literature survey about leading middleware solutions and relating methodology about service selection and composition. Next, we present the problem description and motivations in Section III. The system model we proposed will be discussed with details in Section IV. Implementation details will be described in Section V. Discussion about research findings are presented in Section VI. Finally, we present a conclusion and prospects for future research in Section VII.
